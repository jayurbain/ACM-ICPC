\documentclass{article}

% Macros to make this problem look like the rest of our problems.
\usepackage{icpc_problem}

% Title of your problem.
\title{K: Road Messages}

% Who made the problem
\author{Darko Aleksic}

% Keywords, from a set of standard keywords.
\keywords{simple}

% Anything you want to say about the problem, including how one could solve it
\comments{Picture a block of cheese and a deranged chef with a long straight 
knife stops along two sides.}

% Difficulty on a 1..10 scale.
\difficulty{1}

\begin{document}

\begin{problemDescription}
You were recently hired by Rural and Municipal Roadway Communications to manage 
messages on a scrolling display above a major highway. Much to your surprise, 
these are very primitive displays. You have to input the message manually every 
time it should be changed (there is no memory to preload a list of messages).

Strangely, the only way to post messages is using an on-board stack. You can 
push a character onto the stack, you can pop the top character of the stack, 
and you can print the top character of the stack.  Out of boredom, or perhaps 
the universal human desire to do as little work as possible to get the job done,
you wonder what the minimum number of push, pop, and print are required to print 
a message your boss has told you to display. Oh, you must also ensure the stack 
is clear at the end so that you are ready to input a new message next time your 
boss asks you to do this.

Example: If we want to print the message $abba$ and then clear the stack you 
could do the following. Note the contents of the stack are recorded below with 
the top of the stack on the right.

\begin{verbatim}
operation stack contents displayed-message
1 push a  
2 print a a
3 push b ab a
4 print ab ab
5 print ab abb
6 pop a abb
7 print a abba
8 pop a abba
\end{verbatim}

In fact, this is the fewest operations that can be performed to print exactly 
the message abba and leave the stack empty.

\end{problemDescription}

\begin{inputDescription}
There may be multiple cases to solve.  The first line contains a single 
integer $T <= 30$ indicating the number of test cases. Each of the following
$T$ lines contains a single string consisting of any printable characters. 
The first and last character of each line will not be a space. Each line has 
at least one and at most $200$ characters.
\end{inputDescription}

\begin{outputDescription}
For each case, display the case number followed by the solution, formatted as in 
the sample.  For each of the $T$ strings in the input, you should output on a 
single line the minimum number of operations required to print the string on the display.
\end{outputDescription}

\begin{sampleInput}
4
d
abba
rollover ahead
ogopogo spotted!
\end{sampleInput}
\begin{sampleOutput}
Case 1: 3
Case 2: 8
Case 3: 34
Case 4: 38
\end{sampleOutput}

\end{document}
