\documentclass{article}

% Macros to make this problem look like the rest of our problems.
\usepackage{icpc_problem}

% Title of your problem.
\title{I: Intruder Alert}

% Who made the problem
\author{Charles Riedesel}

% Keywords, from a set of standard keywords.
\keywords{simple}

% Anything you want to say about the problem, including how one could solve it
\comments{comments}

% Difficulty on a 1..10 scale.
\difficulty{1}

\begin{document}

\begin{problemDescription}
A team of secret agents have been seated for a top-secret briefing.  The rows of
chairs are staggered so that attendees can easily interact with the two in
front, the two in back and the one on either side, assuming the attendee is
not along the edge, but notably not the one straight ahead or back two rows.  
We will refer to these potentially six adjacent persons in this honeycomb
cell-like arrangement as neighbors of the secret agent in the middle of the 
cell.  Furthermore the rows are 
arranged such that the chairs on the outside form a convex hull with no gaps.  
The only allowed angles of the sides of the hull allowed by this geometry are 
then 0, 45, and 135 degrees.  All the chairs are filled. See example below.

Unfortunately there is an intruder.  No one really knows anyone else yet, and
each has to depend on whom his/her neighbors claim to be.  The intruder must 
lie about his/her identity, claiming to be one of the secret agents on the
attendance roll.  He/she uses the same false alias when conversing with any
of the neighbors.  Your task is to root out the intruder, or at least the name
of the secret agent he/she is claiming to be.  

It was detected that there is one more person than there were invitees, so
the organizers had each supposed agent, as he/she left in random order after
the meeting, list the up to 6 neighbors. Unfortunately, they did not think
to ask the agent's own name, because if they had, they would have quickly
been able to narrow the intruder down to the two persons with the same name.
\end{problemDescription}

\begin{inputDescription}
Input may consist of multiple cases.  The first line of a case contains a 
positive integer $n$, less than 100, representing the number of invited secret
agents.  Each of the following $n+1$ lines contains the report of an agent
or the imposter.  The number of rows may range from 1 to $n+1$. The report 
consists of six numbers, representing each
neighboring secret agent's claimed identity, such as 007, in clockwise order 
of seating starting at the 3:00 position.  A designation of 000 indicates no 
neighbor, which only happens if the agent is seated along the edge.  Spaces 
delimit the numbers. It is the value of the number that is significant, 
eg. 007 and 07 would indicate the same agent. The numeric values of the agent
designations range from 1 to $n$.  The last case is followed by 
a line containing the number 0 (zero).  The reports are listed in arbitrary 
order, as described by the scenario above, without regard to seating order.
\end{inputDescription}

\begin{outputDescription}
For each case, display the case number followed by the claimed identity, 
formatted as in the sample with 3 digits.   If it is impossible to identify 
the claimed name of the intruder, print the message ``sorry".  Use single 
spaces as delimiters.
\end{outputDescription}

\begin{sampleInput}
13
000  03  02  0  0  0
003  005 04  00 00 001
00   06  05  02 1  00
5  8  7  0  0  2
6  9  8  4  2  3
0 010 09  5  3  00
8  12  11  00  00  04
09 013 12  007 04  05
10  11 13  08  05  6
0  0  11  009  06  0
11 0  0  0  0  007
13 0  0  011  7  8
11  0  0  12  8  9
00 00 00 13  9  10
0
\end{sampleInput}
\begin{sampleOutput}
Case 1: 011
\end{sampleOutput}

\begin{verbatim}
               Example                 Not Valid
                  0                        0
row 1   45   X X X X X X X              X X X X
row 2       X X N N X X X                X X X
row 3      X X N P N X X  45        90  X X X X 90
row 4  135  X X N N X X                  X X X
row 5        X X X X X                  X X X X
                0                          0
\end{verbatim}
The six N's represent the neighbors of agent P.  Angles are shown in degrees.
\end{document}
