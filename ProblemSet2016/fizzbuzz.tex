\documentclass{article}

% Macros to make this problem look like the rest of our problems.
\usepackage{icpc_problem}

% Title of your problem.
\title{E: FizzBuzz}

% Who made the problem
\author{Ryan Patrick}

% Keywords, from a set of standard keywords.
\keywords{simple}

% Anything you want to say about the problem, including how one could solve it
\comments{Picture a block of cheese and a deranged chef with a long straight 
knife stops along two sides.}

% Difficulty on a 1..10 scale.
\difficulty{1}

\begin{document}

\begin{problemDescription}
According to Wikipedia, FizzBuzz is a group word game for children to teach 
them about division. This may or may not be true, but this exercise is
often used to torture screen young computer science graduates during 
programming interviews.  Basically, this is how it works: you print the integers 
from $1$ to $N$, replacing any of them divisible by $X$ with $Fizz$ or, if they are
divisible by $Y$, with $Buzz$. If the number is divisible by both $X$ and $Y$, 
you print $FizzBuzz$ instead.
Check the samples for further clarification.
\end{problemDescription}

\begin{inputDescription}
There may be multiple cases to display.  Each test case will contain three integers 
on a single line: $X$, $Y$ and $N$ $(1 <= X < Y <= N <= 100)$.
The last case is followed by a line containing 0 0 0.
\end{inputDescription}

\begin{outputDescription}
For each case, display the case number followed by the solution, formatted 
as in the sample.  Print integers from $1$ to $N$ in order, each on its own line, 
replacing the ones divisible by $X$ with $Fizz$, the ones divisible by $Y$ with 
$Buzz$ and ones divisible by both $X$ and $Y$ with $FizzBuzz$.
\end{outputDescription}

\begin{sampleInput}
2 3 7
0 0 0
\end{sampleInput}
\begin{sampleOutput}
Case 1: 
1
Fizz
Buzz
Fizz
5
FizzBuzz
7
\end{sampleOutput}

\end{document}
