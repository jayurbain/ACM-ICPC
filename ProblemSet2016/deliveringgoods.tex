\documentclass{article}

% Macros to make this problem look like the rest of our problems.
\usepackage{icpc_problem}

% Title of your problem.
\title{B: Delivering Goods}

% Who made the problem
\author{Darko Aleksic}

% Keywords, from a set of standard keywords.
\keywords{simple}

% Anything you want to say about the problem, including how one could solve it
\comments{thanks!  }

% Difficulty on a 1..10 scale.
\difficulty{1}

\begin{document}

\begin{problemDescription}
You run a delivery company and must deploy a fleet of vehicles to deliver goods 
to clients. All of the goods and delivery trucks are initially located at your 
warehouse.  The road network is given as a directed, weighted graph. One vertex 
is your warehouse and a subset of the remaining vertices are locations of clients 
who will be receiving a package from you today. The weight on an edge represents 
the driving time across that edge.  You guarantee extremely fast shipping: the 
trucks start driving immediately at the start of the day and each client $i$ will 
receive the package at time $t_{i}$ where $t_{i}$ is the driving time of a shortest 
path from the warehouse to their location.

What is the minimum number of trucks you have to deploy to ensure this guarantee 
is met? That is, what is the minimum number of trucks such that it is possible to 
give each truck a driving route so that every client $i$ is visited by some truck 
at time $t_{i}$. Assume it takes no time to load the trucks with the appropriate 
goods at the start of the day, and it takes no time to drop goods off at a client 
once the truck arrives at the client.  These goods are small enough that each truck 
can carry goods for as many clients as necessary.
\end{problemDescription}

\begin{inputDescription}
Input may consist of multiple cases.  The first line of each test case consists of 
three numbers $n$, $m$, and $c$.  Here $n$ denotes the number of vertices in the 
road network $(2 <= n <= 10^{3})$, $m$ denotes the number of edges
$(1 <=  m <= 10^{5})$, and $c$ denotes the number of clients $(1 <= c <= 300, c < n)$.
The vertices are numbered $0$ to $n - 1$. The warehouse is always at vertex $0$. 
The second line consists of $c$ distinct integers between $1$ and $n - 1$
indicating the vertices where the clients reside.

The rest of the input for a case consists of $m$ lines, each containing integers 
$u$,$v$ where $0 <= u,v <= n-1$ and $l$.
This indicates there is a directed edge from $u$ to $v$ with length $l$. Each 
edge's length $l$ satisfies $1 <= l  <= 10^{9}$.
It will always be possible to reach every client from the warehouse.
There will be at most one edge from a vertex $u$ to another vertex $v$, but there
may be edges from both $u$ to $v$ and from $v$ to $u$.

Following the last test case will be a line containing 0 0 0.  There may be blank lines
after each test case for readability.
\end{inputDescription}

\begin{outputDescription}
For each case, display the case number followed by the answer, formatted as in the
sample.  Output a single integer that is the minimum number of vehicles required 
to ensure each client $i$ is visited at time $t_{i}$ by some vehicle.

Explanation: 
In the first sample, one vehicle can follow the path (0,1,2) and the other can 
follow (0,3). In the second sample, the only solution is to use paths (0,1), 
(0,2), and (0,3). In the final sample, one vehicle can follow (0,1), another 
(0,4,6), and the last one (0,2,3,5,7).  
\end{outputDescription}

\begin{sampleInput}
4 5 3
1 2 3
0 1 1
0 3 1
0 2 2
1 2 1
3 2 1

4 5 3
1 2 3
0 1 1
0 3 1
0 2 1
1 2 1
3 2 1

8 11 5
1 3 4 6 7
0 1 5
0 4 1
0 2 2
0 6 6
2 3 1
2 6 3
3 5 7
4 1 5
5 7 3
6 5 6
4 6 4

0 0 0
\end{sampleInput}
\begin{sampleOutput}
Case 1: 2
Case 2: 3
Case 3: 3
\end{sampleOutput}

\end{document}
