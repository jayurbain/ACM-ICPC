\documentclass{article}

% Macros to make this problem look like the rest of our problems.
\usepackage{icpc_problem}

% Title of your problem.
\title{L: Wedgie}

% Who made the problem
\author{Charles Riedesel}

% Keywords, from a set of standard keywords.
\keywords{simple}

% Anything you want to say about the problem, including how one could solve it
\comments{Picture a block of cheese and a deranged chef with a long straight 
knife ready to hack into it.  He takes a couple whacks at the cheese.  Your
task is to determine if the cheese is separated into at least two pieces.
You are given the size of the cheese and three coordinates of the two cuts.
Specifically, the starting point (at a corner or along an edge) and where the
knife stops along two sides.}

% Difficulty on a 1..10 scale.
\difficulty{1}

\begin{document}

\begin{problemDescription}
Picture a rectangular block of cheese sitting on a solid tabletop and a 
deranged chef (yes, just fresh off the range!) with a long straight knife 
ready to hack into it.  He takes a couple whacks at the cheese.  Your
task is to determine if the cheese is separated into at least two pieces.
You are given the size of the cheese and three coordinates of the two cuts.
All measurements are integral and of the same units.  Specifically, the 
points include the starting point (at a corner or along an edge) and where the
knife stops along two sides. Because of the table, no cut starts from the
bottom.  It is possible that the knife goes all the way through, cleaving
the cheese on a single whack.  It is also possible that the two whacks 
intersect, separating out a wedge of cheese.  The tip of the knife is
always outside the block of cheese.

In case the cheese is fully cleaved in a single stroke, the exiting points (always
two, as specified in the problem) will be the two opposite ends of a line (always
at edges or corners) on a face being exited.  You may assume this means the final 
two exit points of the knife from the block.
\end{problemDescription}

\begin{inputDescription}
Input may consist of multiple cases. Each line of a case contains 21 
non-negative integers.  The first 3 are the dimensions of the cheese: 
length, width, and height.  Consider the cheese to be sitting in the 
corner of the first quadrant in 3-D space starting at location 0, 0, 0.
Following this are the descriptions of the two whacks, each consisting of
9 values.  The first 3 of each whack are the coordinates of where the knife 
first enters the cheese along the length, width, and height.  Similarly the 
next two sets of 3 are coordinates for the two ending points somewhere on the 
surface of the block.  An input line showing the cheese block being of size
0, 0, 0 is the sentinel for the end of input.  Do not process it.
\end{inputDescription}

\begin{outputDescription}
For each case, display the case number followed by the word ``touche!" if
the cheese was split or ``must be hard cheese!" if still intact, formatted 
as in the sample.
\end{outputDescription}

\begin{sampleInput}
10 10 10     5 0 10   5 0 5   5 10 5      6 0 10   6 0 5   6 10 5
10 10 10     5 0 10   5 0 5   5 10 5      10 0 6   4 0 6   4 10 6
0 0 0    1 2 3   4 5 6  7 8 9    1 2 3   4 5 6   7 8 9
\end{sampleInput}
\begin{sampleOutput}
Case 1: must be hard cheese!
Case 2: touche!
\end{sampleOutput}

\end{document}
