\documentclass{article}

% Macros to make this problem look like the rest of our problems.
\usepackage{icpc_problem}

% Title of your problem.
\title{C: Enemy of my Enemy is my Friend}

% Who made the problem
\author{Charles Riedesel}

% Keywords, from a set of standard keywords.
\keywords{medium}

% Anything you want to say about the problem, including how one could solve it
\comments{Consider the claim ``The enemy of my enemy is my friend."  Alliances
can become very complicated and intertwined.  Consider a listing of all pairs
of persons, specifying if they are friends or enemies.  Determine the size
of the largest group of mutual friends, assuming the listing is valid
(having no contradictions).}

% Difficulty on a 1..10 scale.
\difficulty{1}

\begin{document}

\begin{problemDescription}
Consider the old maxim ``The enemy of my enemy is my friend".  In difficult times,
alliances can become very convoluted.  Your task is to bring a bit of clarity to
such times, assuming the maxim does describe reality.  You will be provided with
a list of relationships between pairs of persons.  The relationship may be ``friend"
or ``enemy".  There may be additional relationships that are not included in
the list, but these can all be deduced by applying the above maxim.  Your specific
task is to determine if a proposed invitation list includes only mutual friends.
Consider only the reported friends and the friends deduced by using the maxim.
\end{problemDescription}

\begin{inputDescription}
Input may consist of multiple cases.  Each case begins with a line containing 
exactly two integers representing the number of lines of relationship information 
and the number of proposed invitation lists (one per line) respectively for the case.  

Each of the relationship lines begins with either the letter `f' or `e', a number {\em p} 
from 2 to 5, followed by a list of {\em p} names, all delineated by spaces.  In 
case of `f', the names will be of mutual friends.  In case of `e', the names will 
be of mutual enemies.  (Hint - the number of mutual enemies will never be more 
than 2.  Why?)

Each of the proposed invitation list lines begins with a number {\em i} from 2 to 8
followed by {\em i} of the names included in the relationships.
Again, everything is delineated by spaces.

Names will consist of single words of 2 to 10 letters.  Input is case sensitive.
There will be from 1 to 100 names in a case.  The last case is followed by a line
with two 0's (zeroes).

The input will be such that no contradictory relation status will be deducible.
\end{inputDescription}

\begin{outputDescription}
For each case, display the case number followed by the `yes' and `no' answers to 
each of the proposed invitation lists, formatted as in the sample.  A `yes' means
the list is of mutual friends.  A `no' means there is at least one enemy 
relationship or non-existant relationship (a couple that are neither friend or
enemy) in the list.
\end{outputDescription}

\begin{sampleInput}
6  4
e  2  Tom  Fay
f  3  Fay  Mae  Joe
f  2  Al  Mae
e  2  Sam  Tom
e  2  Tom  Al
f  2  Sam  Mae
3  Mae  Fay  Joe
3  Sam  Al  Fay
2  Joe  Al
3  Fay  Sam  Tom
0  0
\end{sampleInput}
\begin{sampleOutput}
Case 1: yes yes no no
\end{sampleOutput}

\end{document}
